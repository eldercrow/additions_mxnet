\documentclass[10pt,twocolumn,letterpaper]{article}

\usepackage{cvpr}
\usepackage{times}
\usepackage{epsfig}
\usepackage{graphicx}
\usepackage{amsmath}
\usepackage{amssymb}

% Include other packages here, before hyperref.

% If you comment hyperref and then uncomment it, you should delete
% egpaper.aux before re-running latex.  (Or just hit 'q' on the first latex
% run, let it finish, and you should be clear).
\usepackage[breaklinks=true,bookmarks=false]{hyperref}

\cvprfinalcopy % *** Uncomment this line for the final submission

\def\cvprPaperID{****} % *** Enter the CVPR Paper ID here
\def\httilde{\mbox{\tt\raisebox{-.5ex}{\symbol{126}}}}

% Pages are numbered in submission mode, and unnumbered in camera-ready
%\ifcvprfinal\pagestyle{empty}\fi
\setcounter{page}{1}
\begin{document}

%%%%%%%%% TITLE
\title{Face Detection Using Single Shot Detector}

\author{Hyunjoon Lee\\
Intel Korea\\
Institution1 address\\
{\tt\small hyunjoon.lee@intel.com}
% For a paper whose authors are all at the same institution,
% omit the following lines up until the closing ``}''.
% Additional authors and addresses can be added with ``\and'',
% just like the second author.
% To save space, use either the email address or home page, not both
%\and
%Second Author\\
%Institution2\\
%First line of institution2 address\\
%{\tt\small secondauthor@i2.org}
}

\maketitle
%\thispagestyle{empty}

%%%%%%%%% ABSTRACT
\begin{abstract}
	A single shot detector (SSD) based face detection algorithm is proposed.
	The proposed detector is efficient, accurate, and robust. 
\end{abstract}

%%%%%%%%% BODY TEXT
\section{Introduction}

With recent advances on deep convolutional neural networks, 
highly accurate face detectors are now available in both academic and industrial field.
Still, there are various challenges remaining in the field, 
such as detection of small faces and real-time performance.
In this paper, a single shot detector (SSD) based face detection algorithm is proposed.
Inheriting efficient structure of the baseline SSD, 
the proposed detector is also capable of detecting faces efficiently.
In addition, a novel network architecture is introduced to detect small faces accurately and robustly.

%-------------------------------------------------------------------------
\section{Related work}
\label{sec:related_work}

Face detectors, generic object detectors, efficient network structures. 
%------------------------------------------------------------------------
\section{Final copy}

You must include your signed IEEE copyright release form when you submit
your finished paper. We MUST have this form before your paper can be
published in the proceedings.


{\small
\bibliographystyle{ieee}
\bibliography{egbib}
}

\end{document}
